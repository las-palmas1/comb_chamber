%%
%% Author: Alexander
%% 12.06.2018
%%

% Preamble
\documentclass[a4paper,10pt]{article}

% Packages
\usepackage{mathtext}
\usepackage[T2A]{fontenc}
\usepackage[utf8]{inputenc}
\usepackage[russian]{babel}
\usepackage{amsmath}
\usepackage{amsfonts}
\usepackage{amssymb}
\usepackage{graphicx}
\usepackage{longtable}
\usepackage[left=2cm,right=2cm,
    top=2cm,bottom=2cm,bindingoffset=0cm]{geometry}
\usepackage{color}
\usepackage{gensymb}

\usepackage{enumitem}
\setlist[enumerate]{label*=\arabic*.}

\usepackage{indentfirst}

\usepackage{titlesec}



% Document
\begin{document}

    \section{Исходные данные.}

    

    
    \begin{enumerate}
        \item Низшая удельная теплота сгорания топлива: $ Q_н = 48.0 \cdot 10^6\ Дж/кг $.
        \item Давление торможения на входе в диффузор: $ p_{вх}^* = 1.5 \cdot 10^6\  Па$.
        \item Температура торможения на входе в диффузор: $ T_{вх}^* = 550\ К $.
        \item Расход на входе в диффузор: $G_{вх} = 45\ кг/с$.
        \item Площадь сечения на входе в диффузор $ F_{вх} = 0.1963\ м^2 $.
        \item Коэффициент полноты сгорания топлива: $eta_{г} = 0.99$.
        \item Скорость на выходе из диффузора: $c_{д} = 60\ м/с$.
        \item Скорость в зоне горения: $c_I = 12\ м/с$.
        \item Теплонапряженность камеры сгорания: $H = 4.5 \cdot10^6\ \frac{Дж}{ч \cdot м^3 \cdot Па}$.
        \item Расход топлива в камере сгорания: $G_т = 0.9\ кг/с$.
        \item Средний диаметр на выходе из диффузора: $D_{д\ ср} = 0.5\ м$.
        \item Суммарный коэффициент избытка воздуха: $\alpha_{\Sigma} = 3.5$.
        \item Коэффициент избытка воздуха в зоне горения: $\alpha_I = 1.1$.
        \item Число жаровых труб: $n = 12$.
        \item Отношение длины зоны горения к длине жаровой трубы: $\frac{l_1}{l_ж} = 0.6$.
        \item Коэффициент сохранения полного давления в диффузоре: $\sigma_д = 0.998$.
        \item Коэффициент сохранения полного давления во фронтовом устройстве: $\sigma_ф = 0.998$.
        \item Скорость на входе в диффузор: $c_{вх} = 24.232\ м/с$.
    \end{enumerate}
    

    \section{Расчет.}

    
    \begin{enumerate}
    	\item Полное давление на выходе из диффузор:
    	\[
    		p_д^* = p_{вх}^* \cdot \sigma_д = 1.5 \cdot 10^6 \cdot 0.998 = 
    		1.497\ Па.
    	\]

    	\item Истинная теплоемкость воздуха на входе в диффузор:
    	\[
    		c_p = 1039.977\ Дж/кг.
    	\]

    	\item Коэффициент адиабаты на входе в диффузор:
    	\[
    		k = 1.382.
    	\]

    	\item Температура торможения на выходе из диффузора:
    	\[
    		T_{д}^* = T_{вх}^* = 550\ К.
    	\]

    	\item Критическая скорость звук на выходе из диффузора:
    	\[
    		a_{кр\ д} = \sqrt{ \frac{2 \cdot k \cdot R \cdot T_{д}^*}{k + 1} } =
    		\sqrt{ \frac{
    				2 \cdot 1.382 \cdot 287.4 \cdot 550
    				}{
    				1.382 + 1
    		} } =
    		428.27\ м/с.
    	\]

    	\item Приведенная скорость на выходе из диффузора:
    	\[
    		\lambda_д = \frac{c_д}{a_{кр\ д}} = \frac{60}{428.27} = 
    		0.14.
    	\]

    	\item ГДФ температуры на выходе из диффузора:
    	\[
    		\tau_д =  1 - \frac{k - 1}{k + 1} \cdot {\lambda_д}^2  =  
    		1 - \frac{1.382 - 1}{1.382 + 1} 
    		\cdot {0.14}^2  = 
    		0.997.
    	\]

    	\item ГДФ давления на выходе из диффузора:
    	\[
    		\pi_д = \left( \tau_д  \right) ^ \frac{k}{k -1} = 
    		\left( 
    			0.997  
    			\right) ^ \frac{1.382}{1.382 -1} = 
    		0.989.
    	\]

    	\item Статическая температура на выходе из диффузора:
    	\[
    		T_{д} = T_{д}^* \cdot \tau_д = 550 \cdot 0.997 = 548.27\ К.
    	\]

    	\item Статическое давление на выходе из диффузора:
    	\[
    		p_{д} = p_д^* \cdot \pi_д = 1.497 \cdot 0.989 = 
    		1.48\ Па.
    	\]

    	\item Статическая плотность на выходе из диффузора:
    	\[
    		\rho_д = \frac{p_д}{R \cdot T_д} = 
    		\frac{1.48 \cdot 10^6}{ 287.4 \cdot 548.27} =
    		9.393\ кг/м^3.
    	\]

    	\item Площадь на выходе из диффузора:
    	\[
    		F_д = \frac{G_{вх}}{ c_д \cdot \rho_д } = \frac{45
    				}{ 
    				60 \cdot 9.393 
    			} =
    		0.0798\ м^2. 
    	\]

    	\item Втулочный диаметр на выходе из диффузора:
    	\[
    		D_{д\ вт} = D_{д\ ср} - \frac{F_д}{\pi \cdot D_{д\ ср}} = 
    		0.5 - \frac{0.0798}{\pi \cdot 0.5} = 
    		0.449\ м.
    	\]

    	\item Периферийный диаметр на выходе из диффузора:
    	\[
    		D_{д\ п} = D_{д\ ср} + \frac{F_д}{\pi \cdot D_{д\ ср}} = 
    		0.5 + \frac{0.0798}{\pi \cdot 0.5} = 
    		0.551\ м.
    	\]

    	\item Полное давление в зоне горения:
    	\[
    		p_I^* = p_д^* \cdot \sigma_ф = 1.497 \cdot 0.998 = 
    		1.494
    	\]

    	\item Критическая скорость звука в зоне горения:
    	\[
    		a_{крI} = a_{кр\ д} =
    		428.27\ м/с.
    	\]

    	\item Приведенная скорость в зоне горения:
    	\[
    		\lambda_I = \frac{c_I}{a_{кр\ д}} = \frac{12}{428.27} = 
    		0.028.
    	\]

    	\item ГДФ температуры в зоне горения:
    	\[
    		\tau_I =  1 - \frac{k - 1}{k + 1} \cdot {\lambda_I}^2  =  
    		1 - \frac{1.382 - 1}{1.382 + 1} 
    		\cdot {0.028}^2  = 
    		1.0.
    	\]

    	\item ГДФ давления в зоне горения:
    	\[
    		\pi_I = \left( \tau_I  \right) ^ \frac{k}{k -1} = 
    		\left( 
    			1.0  
    			\right) ^ \frac{1.382}{1.382 -1} = 
    		1.0.
    	\]

    	\item Статическая температура в зоне горения:
    	\[
    		T_{I} = T_{I}^* \cdot \tau_д = 550 \cdot 1.0 = 549.93\ К.
    	\]

    	\item Статическое давление в зоне горения:
    	\[
    		p_{I} = p_I^* \cdot \pi_I = 1.494 \cdot 1.0 = 
    		1.4933\ Па.
    	\]

    	\item Статическая плотность в зоне горения:
    	\[
    		\rho_I = \frac{p_I}{R \cdot T_I} = 
    		\frac{1.4933 \cdot 10^6}{ 287.4 \cdot 549.93} =
    		9.448\ кг/м^3.
    	\]

    	\item Расход в зоне горения:
    	\[
    		G_I = \frac{ \alpha_I }{ \alpha_\Sigma} \cdot G_{вх} = \frac{1.1}{3.5} 
    		\cdot 45 = 14.143\ кг/с.
    	\]

    	\item Площадь поперечного сечения жаровой трубы:
    	\[
    		F_ж = \frac{ G_I }{n \cdot c_I \cdot \rho_I} = 
    		\frac{
    			14.143
    			}{
    			12 \cdot 12 \cdot 9.448
    		} = 0.0104\ м^2.
    	\]

    	\item Диаметр жаровой трубы:
    	\[
    		d_ж = \sqrt{ \frac{4 \cdot F_ж}{ \pi } } = \sqrt{\frac{4 \cdot 0.0104}{ \pi }} =
    		0.115\ м.
    	\]

    	\item Объем жаровой трубы:
    	\[
    		V_ж = \frac{3600 \cdot Q_н \cdot G_т \cdot \eta_г}{H \cdot p_{вх}^* \cdot n} = 
    		\frac{
    			3600 \cdot 48.0 \cdot 10^6 \cdot 0.9 \cdot 0.99
    				}{
    			4.5 \cdot 10^6 \cdot 1.5 \cdot 12
    		} = 
    		0.0019\ м^3.
    	\]

    	\item Длина жаровой трубы:
    	\[
    		l_ж = \frac{V_ж}{F_ж} = \frac{0.0019}{0.0104} =
    		0.183\ м.
    	\]

    	\item Длина зоны горения:
    	\[
    		l_I = l_ж \cdot \frac{l_1}{l_ж} = 0.183 \cdot 0.6.
    	\]

     \end{enumerate}
    


\end{document}